\documentclass{beamer}
\usetheme[]{Boadilla} 


\setbeamertemplate{blocks}[rounded]

\usepackage{graphicx}
\usepackage{amsmath}
\usepackage{multirow, multicol}
\usepackage{mathpazo}
\usepackage{amsthm}
\usepackage{amssymb}
\usepackage{setspace}
\usepackage{hyperref}
\usepackage{array,colortbl,booktabs}
\usepackage{soul}

\usepackage{enumerate}
\usepackage{url}

\usepackage{verbatimbox}
\usepackage{fancyvrb}
\usepackage{dirtree}


\newcommand{\ds}{\displaystyle}

\newcommand{\bv}{\begin{Verbatim}[numbers=left, baselinestretch=1,
    xleftmargin=.5in, xrightmargin=.1in, frame=single,
    rulecolor=\color{gray}]}



%\usepackage{booktabs}
\title[Intro]{R Introduction}
\author{David Carlson}
\date{September 9, 2016}
\begin{document}
\begin{frame}
\titlepage
\end{frame}

\begin{frame}
\frametitle{$\qquad$\\[-8pt] Exactly What Is This \texttt{R} Thing?}
\begin{itemize}[<+->]
        \item   A \emph{language} and \emph{environment} for statistical computing and graphics.
	    \item	The most powerful and fully featured statistical environment \emph{on the planet}.
        \item   A GNU project based on \texttt{S} which was developed at Bell Laboratories by John Chambers and colleagues.
        \item   Highly extensible through researcher-developed packages and self-developed functions.
        \item   The primary statistical computing tool for academic research statisticians.
        \item   Available for all operating systems.
        \item   Free!
\end{itemize}
\end{frame}

\begin{frame}
\frametitle{What Do You Get?}
\begin{itemize}[<+->]
        \item   An effective data handling and storage facility.
        \item   A suite of operators for calculations on arrays, in particular matrices.
        \item   A large, coherent, integrated collection of intermediate tools for data analysis.
        \item   Graphical facilities for data analysis and display either on-screen or on hard-copy.
        \item   A well-developed, simple and effective programming language.
        \item   Linkages to \texttt{C}, \texttt{Fortran}, \texttt{SQL}, and other languages.
        \item   A huge user/developer community of mostly academics.
\end{itemize}
\end{frame}

\begin{frame}
\frametitle{Major Characteristics}
\begin{itemize}
    \item   Object-Oriented: everything in 	\texttt{R} is an ``object,'' meaning that data structures, functions, created 
            objects are all visible and can be manipulated by name.
    \item   Visible Environment: you see the list of objects you create or download in your 	\texttt{R} environment window.
    \item   Visible Objects: what is \emph{inside} the objects is revealed when you type the name of the object.
    \item   Functions: there are millions of functions in 	\texttt{R} (really!), and you can create your own as well.
    \item   File Coherence: when you quit 	\texttt{R} you will be asked if you want to save your environment (you generally do), and if you say no
            every object created will be gone.
\end{itemize}
\end{frame}

\begin{frame}[fragile]
\frametitle{Getting Help By Yourself}
\begin{itemize}[<+->]
\item StackOverflow (and Google)
	\item	Help pages and FAQ exist on CRAN (\texttt{http://cran.wustl.edu}).
    \item   Local ways to get help in your 	\texttt{R} environment:
            \begin{verbatim}
help(ls);      ?ls
help.search()
args()
View()
browseEnv()
search()
date()
summary()
demo()
            \end{verbatim}
\end{itemize}
\end{frame}

\end{document}
